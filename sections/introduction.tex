Atmospheric aerosols affect climate indirectly by changing cloud properties \citep{lohmann2005global,tao2012impact} and directly by scattering and absorbing sunlight \citep{chylek1995effect}. Among many types of aerosols, soot (or black carbon) is of particular interest, as it strongly absorbs light, thus contributing to global warming, as much as one-third of the contribution of CO\textsubscript{2} \citep{RN21}. Moreover, in many locations worldwide, including major megacities, direct solar heating caused by soot aerosols is comparable with the heating due to greenhouse gases \citep{RN45}. Soot particles suspended in the atmosphere are subject to continuous aging, which changes their microphysical and optical properties, and thus their climatic impacts. Condensation is one of the major aging mechanisms of soot \citep{saathoff2003coating}, and it has been extensively studied using experimental and modeling approaches, including the effects of negative surface curvature \citep{RN70,ivanova2020kinetic} and carbon structure \citep{ivanova2022molecular} on the localization of condensate.

% \textcolor{red}{Ivanova (2022): https://doi.org/10.1021/acs.jced.2c00063}

% \textcolor{red}{Ivanova (2020): https://doi.org/10.1080/02786826.2020.1846677}

% \textcolor{red}{Chen (2018): https://doi.org/10.1021/acs.est.8b04201}

%\citep{RN7,RN4}  \citep{RN51,mikheev2002laboratory}. 


% \textcolor{red}{Zangmeister (2009), Khalizov (2009), Ma (2013), and Mikheev (2002) are not appropriate references} 

% Soot particles are fractal aggregates of graphitic primary spherules and their principal source is the incomplete combustion of carbonaceous matter. Combustion can be used to generate soot aerosols in the laboratory to study their transformations and properties. However, since particle size, organic carbon content, morphology, and concentration of flame-generated soot are highly sensitive to combustion and sampling conditions \citep{RN46,RN47}, researchers often prefer to use pre-made products instead, like carbon black (CB). Such products are manufactured industrially to serve as ink pigments, material additives, etc., making them inexpensive and readily available \citep{RN12}. In spite of a different name, carbon black is inherently the same material as soot. The combustion conditions for CB are much more well controlled to leading to less variability in the generated products. The key difference between CB and flame-generated soot is that the former can be distributed in the form of an aqueous suspension, where aggregates are present as near-spherical particles. In contrast, soot generated and sampled in a flame retains its fractal morphology until the controlled aging process is deliberately initiated, allowing to evaluate changes in optical properties that are induced by not only a coating layer addition, but also by restructuring of the particles.

Advantages of using CB in place of soot include high carbon content, stable number density and size distribution, constant morphology and composition, ease of generation, and safety. Indeed, a simple atomizer that nebulizes a suspension of carbon black particles can generate a stable output over extended time. "Cab-O-Jet" (a brand CB) has been used in a number of studies, including an intercomparison of aerosol absorption cross sections measured with photoacoustic spectroscopy \citep{RN3}, the size and wavelength dependence of optical properties of bare particles \citep{RN4}, and also a comparison of mass absorption spectra obtained at different labs against other surrogates, such as graphene nanoplatelets and fullerene soot \citep{RN63}. These studies concluded that measurements reported by different labs were generally consistent and that surrogates are suitable for instrumental intercomparison. That would likely not be the case with the flame-generated soot, because slight variations in combustion conditions can lead to drastic differences in particle morphology and composition \citep{moore2014mapping}, and hence optics. Another brand of CB, "Regal Black" \citep{RN65}, was used as proxy for collapsed soot to study the response of a Single-Particle Soot Photometer (SP2) to morphology of composite particles produced by coagulation of light absorbing and non-absorbing aerosols, including sodium chloride, ammonium sulfate, and dioctyl sebacate.

Some researchers go as far as to use nigrosin as a surrogate for absorbing aerosols \citep{RN8,RN54,RN55,RN56,RN57,drinovec2022dual}. Nigrosin is a water-soluble organic dye for negative staining of bacteria, which can be nebulized from its solution, and when droplets evaporate upon mixing with dry air, spherical particles are formed. Since nigrosin particles are not fractal, they lack the ability to restructure. Also, being solid spheres, nigrosin particles lack interstitial space that can be present in CB and compacted soot. One factor making nigrosin an attractive surrogate is that extinction and scattering cross sections can be easily calculated for its spherical particles, presumably making experimental results easy to verify computationally. However, when particles are size classified based on their electrical mobility, the resulting aerosol is not truly monodisperse, containing a fraction of multiply charged particles of larger sizes \citep{mcmurry2002relationship,pagels2009processing}. These larger particles have significantly higher absorption and scattering cross-sections, which can bias high optical measurements \citep{RN50} and cause a significant deviation from theoretical predictions. There are ways to account for the presence of multiply-charged particles for both bare and coated aerosols \citep{RN67, RN77}, but the procedures are not always trivial and additional measurements are often required.

This study explores experimentally how flame-generated soot aggregates and their surrogates (commercial CB and nigrosin) respond morphologically and optically to coating by a low-volatility organic compound, dioctyl sebacate (DOS). For CB, we report measurements for both compact aggregates and agglomerates of compact aggregates. Corrections for multiple charging are applied and their efficacy is analyzed and discussed for different particle types. Finally, the optical properties of soot, two forms of CB, and nigrosin are compared against each other and also against three commonly used optical models, to determine applicability ranges for the optical models and soot surrogates.
